
\documentclass[10pt,a4paper,twoside]{article}
\usepackage[dutch]{babel}
\usepackage{graphicx}
\usepackage{float,flafter}
\usepackage{hyperref}
\usepackage{inputenc}
%zet de bladspiegel :
\setlength\paperwidth{20.999cm}\setlength\paperheight{29.699cm}\setlength\voffset{-1in}\setlength\hoffset{-1in}\setlength\topmargin{1.499cm}\setlength\headheight{12pt}\setlength\headsep{0cm}\setlength\footskip{1.131cm}\setlength\textheight{25cm}\setlength\oddsidemargin{2.499cm}\setlength\textwidth{15.999cm}

\begin{document}
\begin{center}
\hrule

\vspace{.3cm}
{\bf {\Large Assignment 4 }}\\
{\bf {\huge Quantum Teleportation}}
\vspace{.2cm}
\end{center}
{\bf Name:}  Priyanshu Gupta\\
{\bf Roll no:}  19111042 \\
{\bf Branch: }  Biomedical Engineering \hspace{\fill}  3 August, 2021 \\
\hrule

\vspace{.4cm}
\begin{center}{\textbf{\large Can AI be used to understand facts about the topic and bring that topics from "pseudoscience" to "science".}} \end{center}

\textbf{Teleportation} is the hypothetical transfer of matter or energy from one point to another without traversing the physical space between them. This phenomenon is not allowed in the physical real world until present day but could be possible at subatomic level through the transfer of quantum information or technically quantum computing. \\

In the quantum world, important point to keep in mind is that quantum teleportation only transfers quantum information or quantum states, rather than the transportation of matter, from one location to another using quantum entanglement, where the two particles in separate locations are connected by an invisible force or the spooky action also known as quantum entanglement. In entanglement, the properties of one particle affect the properties of another, even when the particles are separated by a large distance. Quantum teleportation involves two distant, entangled particles in which the state of a third particle instantly teleports its state to the two entangled particles.\\

One of the basic problems in quantum computing is finding a fast and reliable method to move a quantum bit (or qubit) – the basic piece of quantum information in the machine. This piece of information is coded by a single electron that has to be moved between two positions without passing through any of the space in between. This problem does has a solution but it is a purely quantum process and the method applies only in ideal conditions, when the electron state suffers no disturbances or perturbations.\\

Then here comes the AI approach. Some Italian researchers have shown that it is possible to teleport a qubit in what might be called a real-world situation by letting artificial intelligence do much of the thinking. It could be done using the concept of \textbf{deep learning’s artificial intelligence algorithms} and applying it to the field of quantum computers. The technique used is called as \textbf{"reinforcement learning"} where the agent percepts from its environment and acts in order to maximize the reward function with optimal efforts.\\

 This techniques is based on artificial neural networks arranged in different layers, each of which calculates the values for the next one so that the information is processed more and more completely. The setup of this artificial intelligence method could be assigned with the task of discovering how to control the qubit alone and find its own fast solution such that it can adapt the disturbances and perturbations. The artificial intelligence system learns to create a variety of entangled states and improves the efficiency of their realization.\\

In other words the use of artificial intelligence in design and control of quantum computers could help us in solving more and more problems and we would be able to gain more knowledge from this field. The research is an important step in improving quantum computing and has the potential to revolutionize technology, medicine and science by providing processors and sensors faster and more efficient than present day AI algorithms. Quantum computing can provide a computation boost to artificial intelligence, enabling it to tackle more complex problems.


\end{document}
